\documentclass[letter,11pt]{article}
%\special{papersize=215mm,297mm}
\usepackage[spanish]{babel}
\usepackage[utf8]{inputenc}
\usepackage[american]{circuitikz} 
\usepackage{graphicx}
\usepackage{color}
\usepackage{calendar}
%\usepackage{lscape}
\usepackage[margin=1in]{geometry}
\usepackage{hyperref}

\newcolumntype{P}[1]{>{\centering\arraybackslash}m{#1}}
\renewcommand{\arraystretch}{1.5} % Increase vertical row height for better centering
\usepackage{array} % Supports fixed column widths with `p{}`
\usepackage{geometry} % For setting page margins
\usepackage{graphicx} % For `\resizebox` functionality\usepackage{amsmath}
\usepackage{amssymb}
\usepackage{amsfonts}
\usepackage{makecell}


\hyphenpenalty=10000000
\usepackage{url}
\usepackage{fancyhdr}
\usepackage{float}
\usepackage{wrapfig}
\usepackage{enumerate}


\setlength\textwidth{7in}
\oddsidemargin=-0.5cm


\pagestyle{fancy}
\renewcommand{\headrulewidth}{0.1pt}
\renewcommand{\footrulewidth}{0.1pt}
\headheight=-5pt
\lhead{\scshape Pontificia Universidad Cat\'olica de Chile}
\rhead{\scshape IIBM}
\rfoot{}
\lfoot{}
\cfoot{\thepage}

\setlength{\parindent}{0in}
\setlength{\parskip}{0.1in}


\begin{document}

	\begin{flushleft}\small
		\large{\textbf{IIBM BootCamp 2025 \href{https://github.com/cgvalle/IIBM-BootCamp-2025}{GitHub}}}\\
		\textbf{Instructors: } Carlos Valle (\href{mailto:cgvalle@uc.cl}{cgvalle@uc.cl})  
		\hspace{-0.37cm} \begin{tabular}{llllll}

\end{tabular}
\end{flushleft}





\begin{center}
\LARGE\textbf{IIBM Bootcamp 2025}\\
\end{center}



\large
This course, both theoretical and practical, aims to equip students with essential programming skills and mathematical methods for addressing challenges in medicine and biology. The primary goal of this Bootcamp is to prepare students for the postgraduate program of Institute for Biological and Medical Engineering (IIBM) by providing them with the necessary tools for success in various program courses.

Participants will learn the fundamentals of Python and MATLAB, programming languages widely used in scientific computing, along with basic concepts in calculus, linear algebra, and an introduction to image processing. Faculty and graduate students from the IIBM will guide students through interdisciplinary problem-solving, emphasizing the importance of programming and computational tools.

\vspace{0.3cm}

\underline{\textbf{Learning objectives}}:
\begin{itemize}
    \itemsep0em
    \item Acquire basic programming skills
    \item Apply programming tools
    \item Design basic scripts
    \item Explain chosen methods and obtained results to an interdisciplinary and diverse audience
    \item Contrast the results critically and respectfully with different people
    
\end{itemize}

\underline{\textbf{Day scheme}}:

The Bootcamp will be held from \textbf{January 13 to 17 from 09:00am to 17:00pm}. The scheme for most days is:

\begin{itemize}
    \itemsep-0.3em 
    \item \textbf{09:00 - 10:10} \Large Introduction and Hands-on Coding \large
    \item \textbf{10:10 - 10:30} \Large Break \large
    \item \textbf{10:30 - 12:30} \Large Hands-on Coding \large
    \item \textbf{12:30 - 13:30} \Large Guest professor\large
    \item \textbf{13:30 - 14:30} \Large Lunch \large
    \item \textbf{14:30 - 16:30} \Large Group project (pairs) \large
    \item \textbf{16:30 - 17:00} \Large Presentations and discussion \large
    
\end{itemize}

On the first day, we will meet at \textbf{08:45 AM} at the Institute for Biological and Medical Engineering, located on the 7th floor of the Ciencia y Tecnología Building at Campus San Joaquín UC. \textbf{A laptop with internet connection, Python and Matlab will be required for the Bootcamp.} 



%% New page %%
\newpage
\underline{\textbf{Bootcamp Topics}}:

\vspace{0.3cm}
\hspace{0.7cm}  \Large \textbf{Introduction} \large
\vspace{-0.5cm}
\begin{enumerate}
    \itemsep-0.4em 
    \item Overview of the course
    \begin{itemize}
        \vspace{-0.3cm}
        \item Course description and objectives
        \item State of the art in coding and IIBM project examples
    \end{itemize}
    \item Setting up Tools
        \begin{itemize}
            \vspace{-0.3cm}
            \item Google Colab
            \item Github
        \end{itemize}
        
\vspace{0.3cm}
\Large \textbf{Programming skills} \large
    \item Basic data types: Strings, lists, numbers (int and float) and booleans
    \item Control process:
    \begin{itemize}
        \vspace{-0.3cm}
        \item Loops: for and while
        \item Conditions and if statements
        \item Control statements: break, continue and pass
    \end{itemize}
    \item Arithmetic operators and naming conventions
    \begin{itemize}
        \vspace{-0.3cm}
        \item Arithmetics operators (/, //, \%, **, + and -)
        \item Naming conventions for variables and functions
    \end{itemize}
    \item Functions and scripts
    \item Data reading
    \begin{itemize}
        \vspace{-0.3cm}
        \item Reading from files (.txt and .csv)
        \item Reading images
        \item Common libraries for data reading
    \end{itemize}
    \item Introduction to Numpy and Matplotlib libraries
    \begin{itemize}
        \vspace{-0.3cm}
        \item Numpy: Operating with Matrix and vectors
        \item Matplotlib: Plots and parameters 
    \end{itemize}
    \item Debugging: Python and Matlab
    
\vspace{0.2cm}
\Large \textbf{Calculus and Algebra} \large
    \item Calculus
    \begin{itemize}
        \vspace{-0.3cm}
        \item Derivatives
        \item Integrals
    \end{itemize}
    \item Algebra
    \begin{itemize}
        \vspace{-0.3cm}
        \item Matrix operations
        \item Vector operations
    \end{itemize}

\vspace{0.2cm}
\Large \textbf{Image processing} \large
    \item Basic operations
    \begin{itemize}
        \vspace{-0.3cm}
        \item Filter
        \item Fourier transform
    \end{itemize}

\end{enumerate}


\underline{\textbf{Week schedule}}:

\pagestyle{empty} % Disable default headers and footers

\setlength{\parindent}{0pt} % Stop paragraph indentation

\StartingDayNumber=2 % Calendar starting day, default of 1 means Sunday, 2 for Monday, etc

%----------------------------------------------------------------------------------------
%	TITLE SECTION
%----------------------------------------------------------------------------------------
%\begin{center}
%	\textsc{\LARGE University Timetable}\\ % Title text
%	\textsc{\large Semester 1} % Subtitle text
%\end{center}
%----------------------------------------------------------------------------------------
Location: \textbf{TBA}
\Large

\resizebox{\textwidth}{!}{ % Scale table to fit text width
\begin{tabular}{|P{1.5cm}|P{3.5cm}|P{3.5cm}|P{3.5cm}|P{3.5cm}|P{3.5cm}|} % Table structure - 6 columns (Time + 5 days)
\hline
\textbf{Time} & \multicolumn{1}{c|}{\textbf{Monday}} & \multicolumn{1}{c|}{\textbf{Tuesday}} & \multicolumn{1}{c|}{\textbf{Wednesday}} & \multicolumn{1}{c|}{\textbf{Thursday}} & \multicolumn{1}{c|}{\textbf{Friday}} \\
\hline
\hline
\textbf{Room} & \multicolumn{5}{|c|}{\textbf{TBA}} \\

\hline
09:00 \ 10:10 & Working with Google Colab & Python Arithmetics  & Python Matrix and Plots  & Derivatives Part 1  & Linear Systems  \\
\hline
10:10 \ 10:30 & \multicolumn{5}{|c|}{\textbf{Break}} \\ % "Break" row spanning all 5 days
\hline
10:30 \ 12:30 & Intro to Variable Types  & Python Functions  & Matlab Introduction  & Derivatives Part 2  & Matlab Images  \\
\hline
12:30 \ 13:30 & Speaker: \textbf{Rene Botnar} & Speaker: \textbf{Tobias Wenzel} & Speaker: \textbf{Flavia Zaccon} & Speaker: \textbf{Pablo Irarrázaval } & Speaker: \textbf{TBA} \\
\hline
13:30 \ 14:30 & \multicolumn{5}{|c|}{Lunch}\\
\hline
14:30 \ 16:30 & Control Flow  & \multicolumn{4}{|c|}{Group Project} \\
\hline
16:30 \ 17:00 & Discussion and Content Check & \multicolumn{3}{|c|}{Group Project Presentation}& Group Project Presentation and Final Thoughts \\
\hline
\end{tabular}
} % End of resizebox




\end{document}



